% !TeX encoding = UTF-8
% !TeX root = MAIN.tex

\newif\ifeng%
%% HINWEISE: Hier müssen folgende Einstellungen vorgenommen werden:
%% PLEASE NOTE: Select your settings here:

%% Sprache: Falls die Dokumentensprache Deutsch ist, \engtrue mit einem %-Zeichen davor auskommentieren:
%% Language: If the document language is German, comment \engtrue with a % sign in front:
\engtrue%

%% Hier den Namen des Autors eingeben:
%% Enter the author’s name here:
\def\author{Philipp Renz}

%% Hier Informationen für den rechten Block unter dem JKU-Logo eingeben, wobei die Elemente mit einem Buchstaben jeweils für die Beschreibung und mit Doppelbuchstaben für den Inhalt sind.
%% Anzuführen bei Masterarbeit: Eingereicht von, Anfegertigt am, BeurteilerIn, Mitbetreuung.
%% Anzuführen bei Dissertation: Eingereicht von, Anfegertigt am, ErstbeurteilerIn, ZweitbeurteilerIn, Mitbetreuung.
%% Anzuführen bei strukturiertem Doktorat: Eingereicht von, Angefertigt am, ErstbetreuerIn, ZweitbetreuerIn, Mitbetreuung.
%%
%% Enter information here for the right block under the JKU logo, whereby the elements should have one letter for the heading and double letters for content.
%% To be given for master thesis: Author, Submission, Thesis Supervisor, Assistant Thesis Supervisor.
%% To be given for doctoral thesis: Author, Submission, First Supervisor, Second Supervisor, Assistant Thesis Supervisor.
\def\elementA{Submitted by}
\def\elementAA{\textbf{\author} \\ 01126686}

\def\elementB{Submitted at}
\def\elementBB{\textbf{Institute for Machine Learning}}

\def\elementC{Thesis Supervisor / First Evaluator}
\def\elementCC{Univ.-Prof.~Mag.~Dr.\@ \textbf{Günter Klambauer}}

\def\elementD{Co-Supervisor}
\def\elementDD{Univ.-Prof. Dr. \textbf{Sepp Hochreiter}}

\def\elementE{Second Evaluator}
\def\elementEE{Prof. Dr. \textbf{Andreas Bender}}

%% Hier Datum eingeben (Monat der Abgabe im Prüfungs- und Anerkennungsservice):
%% Enter the date (Month and year of submission to Examination and Recognition Services):
\def\date{September 2024}

%% Hier Ort eingeben:
%% Enter the location:
\def\place{Linz}

%% Hier Titel eingeben; steht über dem K:
%% Enter the title; it appears above the K:
% \def\title{Advancing Evaluation of Generative Models for Molecules and Deep Learning for Reaction Prediction}
% \def\title{Advancing Molecular Generative Models and Retrosynthesis Prediction: New Evaluation Approaches and Deep Learning Techniques for Drug Discovery}
\def\title{Generative Models in Drug Discovery: \Huge Advancing Assessments, Metrics and Retrosynthesis Prediction}
% \def\title{Advancing Evaluation of Molecule Generators and Deep Learning for Retrosynthesis Prediction}

%% Hier den Typ der Arbeit eingeben (0: Keine Arbeit, 1: Bachelorarbeit, 2: Masterarbeit, 3: Dissertation, 4: Diplomarbeit):
%% Enter the type of paper here (0: Not Thesis, 1: Bachelor’s Thesis, 2: Master’s Thesis, 3: Dissertation, 4: Diploma Degree Thesis):
\def\type{3}

%% Hier ggf. Untertitel eingeben; stehen unter dem K (nur bei 0):
%% If necessary, enter a subtitle here; below the K (only for 0):
\def\subtitle{}

%% Hier den angestrebten akademischen Grad eingeben:
%% Enter the desired academic degree here:
\def\acadDegree{Doktor der Naturwissenschaften}

%% Hier die Studienrichtung eingeben:
%% Enter the major here:
\def\study{Naturwissenschaften}

% Clean thesis data

% !TEX root = my-thesis.tex

% **************************************************
% Files' Character Encoding
% **************************************************
\PassOptionsToPackage{utf8}{inputenc}
\usepackage{inputenc}

% **************************************************
% Load and Configure Packages
% **************************************************
\usepackage[english]{babel} % babel system, adjust the language of the content
\PassOptionsToPackage{% setup clean thesis style
    figuresep=colon,%
    hangfigurecaption=false,%
    hangsection=false,%
    hangsubsection=false,%
    sansserif=false,%
    configurelistings=true,%
    colorize=full,%
    colortheme=bluemagenta,%
    bibfile=references_zotero,%
}{cleanthesis}
\usepackage{cleanthesis}



\hypersetup{% setup the hyperref-package options
    pdftitle={Generative Models in Drug Discovery: Advancing Assessments, Metrics and Retrosynthesis Prediction},    %   - title (PDF meta)
    pdfsubject={Documentation},%   - subject (PDF meta)
    pdfauthor={Philipp renz},    %   - author (PDF meta)
    plainpages=false,           %   -
    colorlinks=false,           %   - colorize links?
    pdfborder={0 0 0},          %   -
    breaklinks=true,            %   - allow line break inside links
    bookmarksnumbered=true,     %
    bookmarksopen=true          %
}


