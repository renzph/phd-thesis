\chapter{Introduction\label{chap:introduction}}
% Drug discovery is important and ML can help
Drug discovery is the process of identifying and bringing to market new
medicines and our quality of life greatly depends on it. The use of machine
learning (ML) and deep learning (DL) has the potential to accelerate the drug
discovery process. Advances in ML and DL have recently led to breakthroughs
in many areas, and led to a surge in interest in applying ML and DL to drug
discovery.

% Identification of drug candidates is expensive. ML can help
One of the key challenges of the drug discovery process is the identification
of novel promising drug candidates. Traditionally, this is done by synthesizing
molecules in a laboratory and testing them for their efficacy against a target.
However, this process is very expensive and time-consuming. Using ML and DL
to identify promising drug candidates has the potential to accelerate the
drug discovery process by first identifying promising candidates in a virtual
environment.

% Generative models help to find promising candidates
Generative models for molecules aim to address the challenge, of finding novel
molecular structures satisfying a property profile of interest. Given a scoring
function that encodes these properties, generative models create molecular
structures in a feedback loop, where the scoring function guides the generation
process. In contrast to the traditional approach of brute-force searching a
fixed virtual library of molecules, generative models are able
navigate the chemical space more efficiently.

% The evaluation of generative models is challenging. 
However, the evaluation of generative models is challenging. While
discriminative models have non-ambiguous prediction targets, that can easily be
evaluated, using relevant hold-out datasets, the evaluation of generative models
differs by some key aspects. Firstly, prediction tasks have many possible
solutions. In general it is favourable for a generative model to generate a
diverse set of high-scoring molecules. Secondly, they should be able to be
``inventive'' and generate solutions that are considered sufficiently succinct
from the training data used to inform the scoring function. Thirdly, the
performance of generative models depends heavily on available compute budget.
While the performance of discriminative models in drug discovery is often
limited by the small amount available data, generative models can effectively do
an exhaustive search of the chemical space, given a large enough compute budget.

% Retrosynthesis planning is important and ML can help
Synthesis planning is another crucial step in drug discovery.
In order to be able to actually synthesize a molecule, a chemist needs to
identify a sequence of chemical reactions that transform a set of available starting
materials into the target molecule. Camputer-aided synthesis planning (CASP)
aims to assist chemists in this task. CASP usually combine a planning method
with a computational model of chemical reactions. The quality of the
synthesis plan depends heavily on the accuracy of the used reaction model.

% A prominent approach to CASP is template-based planning
A prominent approach to CASP is template-based planning. In this approach
chemical reactions are modelled graph transformation rules, templates, that
encode connectivity changes between atoms during a chemical reaction.
This effectively reduces the reaction model to a classification problem, where
the goal is to predict which templates can be used to synthesize a target molecule.
However, this results in a large number of classes, with many classes only
represented by few training samples. This makes it difficult to train a model
that generalizes well for rare templates.

\paragraph{Do generative models exhibit biases towards the used scoring
    functions and training data, and how can we effectively detect this?} The
scoring functions optimized by generative models are often statistical
machine learning models, which usually exhibit biases towards the training
data and model the data distribution in a unique way that depends on the
random initialization of the model. A generative model that overfits to
these biases will be less useful in practice as it is likely to generate
molecules that are similar to the training data. Furthermore, reported
performance metrics may be misleading and give an overoptimistic estimate of
the performance of the generative model. To clarify this issue we proposed a
set of control scores that give us insights whether generative models adapt
to the biases of the scoring function and training data.
\Cref{sec:overview-failure-modes} describes our findings in more detail and
\cref{sec:failure-modes} reprints the corresponding publication.

\paragraph{How do generative models compare in finding diverse high-scoring molecules?}
Meaningfully comparing generative models in diverse optimization settings is challenging.
Firstly, there is no consensus on how to best measure diversity and commonly used
diversity metrics are not relevant for the problem at hand. Secondly, a meaningful
comparison of optimization algorithms necessitates a standardized compute budget.
To address the issue of a lacking, meaningful benchmark we define a
benchmark for diverse optimization that considers the above-mentioned issues.
\Cref{sec:overview-diverse-efficiency} gives an overview over this topic
and the corresponding publication is reprinted in \cref{sec:diverse-efficiency}.

\paragraph{Can we improve the performance of reaction modelling using modern DL approaches?}
In \cref{sec:overview-mhn-react} we describe our work on improving the performance
of reaction modelling using a multimodal learning approach based on modern
Hopfield networks. The idea behind this approach is to learn relevant representations
of molecules and reaction templates and to associate them using a Hopfield network.
This allows for making use of similarities between templates and generalizing over
different templates instead of modelling them as distinct categories, leading to
improved performance, especially for templates with few training samples and even
for templates unseen during training.
The corresponding publication is reprinted in \cref{sec:mhn-react}.

\section{List of Papers}
This thesis comprises the work published in the following papers:

\begin{itemize}
    \item \fullcite{renzFailureModesMolecule2019}
    \item \fullcite{renzBenchmarkingEfficiencyGenerative2024}
    \item \fullcite{seidlImprovingFewZeroShot2022}
\end{itemize}

% give overview over my other publications
\paragraph{Other Publications} Besides the papers listed above, I have also
contributed to the following publications:

\begin{itemize}
    \item \fullcite{preuerFrechetChemNetDistance2018}
    \item \fullcite{renzUncertaintyEstimationMethods2019}
    \item \fullcite{hofmarcherLargescaleLigandbasedVirtual2020}
    \item \fullcite{renzLowCountTimeSeries2023}
\end{itemize}

\section{Overview}
The following provides an overview over the publications that are part of this
thesis.

\subsection{On Failure Modes in Molecule Generation and Optimization}
Generative models for molecules are often categorized into two groups:
\emph{distribution-learning} and \emph{goal-directed} models.
Distribution-learning models are trained to learn the distribution of
molecules in a dataset, and can be used to generate novel molecules.
Goal-directed models are trained to optimize a scoring function, and can be
used to generate molecules with desired properties.

A common distribution-learning benchmark is the \emph{GuacaMol} benchmark
\citep{brownGuacaMolBenchmarkingModels2019}. In \citep{renzFailureModesMolecule2019}
we show how this benchmark is not able to distinguish between sophisticated
generative models and a simple baseline model that generates molecules
by introducing minor variations of known molecules.

For goal-directed models, we study the optimization of scoring functions that are
based on machine learning models. We introduce control scores that give us information
about whether the optimization overfits to artifacts of the scoring function and
whether molecule generation is biased towards the training data.

\subsection{Evaluation of Models for Diverse Molecule Generation\label{sec:overview-diverse-efficiency}}
Generative models for molecules are most often evaluated by the average score of
the top scoring molecules. However, this metric is not suited when the goal is to
generate a wide variety of molecules, as is often the case in early stage drug
discovery. However, little is known about how well different approaches to
diverse optimization perform in practice, which is due to three main reasons.
Firstly, there is no consensus on how to measure diversity and commonly used
diversity metrics are not suited for evaluating diverse optimization approaches.
Secondly, a meaningful comparison of optimization algorithms necessitates a
standardized compute budget. Thirdly, there exists no benchmark that compares
a wide range of diverse optimization approaches.

In \citep{renzBenchmarkingEfficiencyGenerative2024} we address these issues by
proposing a benchmark for diverse optimization that addresses the above-mentioned
issues. Firstly, we make use of a recently proposed diversity metric that is
suited for evaluating diverse optimization approaches. Secondly, we evaluate
the methods under two different settings. In the first setting, the methods are
compared under a fixed number of scoring function evaluations, which is relevant
for applications where the scoring function is expensive to evaluate. Generative
models are only useful if they are sample efficient and reach a high score with a
minimal number of scoring function evaluations. In the second setting, the methods
are compared using a time budget, which is relevant for scoring functions that are
cheap to evaluate. In this setting, the methods should be computationally efficient.

In our paper we establish a ranking of diverse optimization approaches under both settings.
We find that it depends on the used compute budget which method performs best.
Furthermore, we find that some methods performing well for single-molecule optimization
perform less well for diverse optimization.

\subsection{Single-step Retrosynthesis Prediction\label{sec:overview-mhn-react}}
% In a template-based approach, reaction templates are first
% extracted from a reaction database or hand-coded by a chemist.
% If the product side of a template is a subgraph of a molecule,
% the template is called applicable to the molecule and can be
% used to transform it to a reactant set. However, even if a
% template can be applied to a molecule, the resulting reaction
% might not be viable in the laboratory.11 Hence, a core task,
% which we refer to as template-relevance prediction, in such an
% approach is to predict with which templates a molecule can be
% combined with to yield a viable reaction. In prior work, this
% problem has often been tackled using machine learning
% methods that are trained at this task on a set of recorded
% reactions

% Template-based methods usually view the problem as a
% classification task in which the templates are modeled as
% distinct categories. However, this can be problematic as
% automatic template extraction leads to many templates that are
% represented by few training samples,9,26 Somnath et al.25
% argued that template-based approaches suffer from bad
% performance, particularly for rare reaction templates. Segler9
% and Struble et al.10 noted that machine learning (ML) has not
% been applied successfully for CASP in low-data regimes. To
% address the low-data issue, Fortunato et al.26 pretrained their
% template-relevance model to predict which templates are
% applicable and then fine-tuned it on recorded reactions in a
% database. This improved template-relevance prediction,
% especially for rare templates, as well as the average applicability
% of the top-ranked templates. Overall, a challenge of template-
% based methods arises from modeling reaction templates as
% distinct categories, which leads to many classes with few
% examples (see the section entitled “Methods”).
% To avoid the above-mentioned problems, we propose a new
% model that does not consider templates as distinct categories,
% but can leverage structural information about the template.
% This allows for generalization over templates and improves
% performance in the tasks defined in ref 26, especially for
% templates with few training samples and even for unseen
% templates. This model learns to associate relevant templates to
% product molecules using a modern Hopfield network
% (MHN).33,34 To this end, we adapted MHNs to associate
% objects of different modalities, namely input molecules and
% reaction templates. A depiction of our approach is illustrated in
% Figure 1.
% In contrast to popular ML approaches, in which variable or
% input-dependent subsets of the data are associated,22,33,35,36
% our architecture maintains a fixed set of representations,
% considered as a static memory independent of the input.
% In this study, we propose a template-based method, which
% are often reported to be computationally expensive, because of
% the NP-complete subgraph-isomorphism calculations involved
% in template execution.24−26,28 To address this issue Fortunato
% et al.,26 Bjerrum et al.28 trained neural networks to predict
% which templates are applicable, given a molecule to filter
% inapplicable templates during inference. We find that using a
% substructure screen, i.e., a fast check of a necessary condition
% for a graph to be a subgraph of another improves inference
% speed, which may also be of interest for other template-based
% methods.
% The main advance of our model over Fortunato et al.,26
% Hasic and Ishida,37 or other template-based methods, is that by
% representing and encoding reaction templates we are able to
% predict relevant templates, even if few training data is available,
% which is a common issue in reaction datasets.
% This work is structured as follows: In the “Methods” section,
% we propose a template relevance model that predicts template
% relevance by applying a multimodal learning approach using a
% modern Hopfield network. In the sections entitled “Template
% Prediction” and “Single-Step Retrosynthesis”, we demonstrate
% that our architecture improves predictive performance for
% template relevance prediction and single-step retrosynthesis. In
% the section enetitled “Inference Speed”, we show that our
% method is several times faster than baseline methods.

In \citep{seidlImprovingFewZeroShot2022} we study the problem of single-step
retrosynthesis prediction. Given a target molecule, the goal is to predict a
set of reactants that can be used to synthesize the target molecule.
Template-based models are a popular approach to single-step retrosynthesis
prediction. These models use so called reaction templates, which are
either extracted from a reaction database or hand-coded by a chemist.
Given a target molecule, the goal is then reframed as a classification task,
where the model predicts which templates can be used to synthesize the target.

While previous work has modelled the templates as distinct categories, we
propose a model that can leverage structural information about the templates.
For this we use a Modern Hopfield Network \citep{ramsauerHopfieldNetworksAll2020}, which
learns to associate relevant templates to product molecules.
This allows for generalization over templates and improves performance, especially
for templates with few training samples and even for unseen templates.

We could show that our models showed state-of-the-art performance in
the single-step retrosynthesis prediction task, and that our model is several
times faster than other methods.