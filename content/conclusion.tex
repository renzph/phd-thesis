\chapter{Conclusion and Outlook\label{chap:conclusion}} The work in this thesis has focused on
advancing the application of generative models in drug discovery, concentrating on two main aspects:
Firstly, we identified limitations in the evaluation of generative models for de novo molecular
design, and proposed ways to make evaluation more informative and relevant to practical
applications. Secondly we introduced a novel template-based model for retrosynthesis prediction that
exceeds the performance of existing methods, performing particularly well on rare reaction
templates.

In the first part of this thesis, we showed how established an established benchmark for
distribution-learning models cannot differentiate complex models from trivial baseline generators,
highlighting the need for more informative evaluation metrics. Furthermore, we showed how
goal-directed generative models can overfit to machine learning based scoring functions and exhibit
biases towards the training data. We proposed control scores as a diagnostic tool to identify
overfitting and biases in goal-directed molecular generation.

The second part of this thesis introduced a diversity-based benchmark for goal-directed molecule
generators aiming at measuring their performance in generating diverse high-scoring molecules.
This benchmark addresses the shortcomings of previous benchmarks by adressing the issues of
inadequate diversity measures, non-standardized compute budgets, and lack of model adaptation to the
diverse optimization setting. We used this benchmark to evaluate a range of generative models

The third and last part of this thesis introduced a novel template-based model for retrosynthesis prediction
based on Modern Hopfield Networks. This model leverages a multi-modal approach that combines
reaction templates and target molecules. Our model reaches state-of-the-art performance while
maintaining a significantly lower computational cost compared to existing methods. The model is able
to generalize over reaction templates and performs particularly well on rare reaction templates.comparing them in a meaningful way, finding that SMILES-based autoregressive models perform best.

The control scores have since been further investigated in \citep{turkMolecularAssaysSimulator2022}.
We join them in their emphasis that while the control scores can identify the presence of biases,
they cannot establish their absence as the optimization and control scores might be highly
correlated even outside the applicability domain of the models. \citet{thomasComparisonStructureLigandbased2021}
showed that optimizing structure-based scoring functions leads to more diverse and novel molecules
compared to ligand-based scoring functions. This further supports the bias towards active compounds in the
training data when optimizing ligand-based scoring functions.

Going forward we see more comprehensive benchmarks for goal-directed generators as a key aspect to
advance the field of de novo molecular design. Despite significant efforts
\citep{brownGuacaMolBenchmarkingModels2019,gaoSampleEfficiencyMatters2022,gaoSynthesizabilityMoleculesProposed2020,thomasMolScoreScoringEvaluation2024}
towards this end, there is a lack of unifying these efforts. A key challenge is the establishment of
relevant scoring functions that better reflect the difficulties in real-world drug discovery
projects \citep{fromerComputeraidedMultiobjectiveOptimization2023} without suffering from the
problems observed in \Cref{sec:failure-modes} and \citep{lyuModelingExpansionVirtual2023}. These
benchmarks should also better incorporate whether the chemistry of the generated molecules is
reasonable \citep{thomasReevaluatingSampleEfficiency2022} and diverse. Furthermore the benchmarks
should be run at scales that are more comparable to real-world drug discovery projects, given that
relative performance of the models can be influenced by the available compute budget. Finally, we
think that evaluating synthesizability of the generated molecules is a crucial aspect, which is a
challenge by itself. While being a significant effort, we believe the establishment of such a
benchmark would spur the development of more practically relevant generative models.

In the field of retrosynthesis prediction, we also think that there is a need for more better
evaluation strategies. Efforts in this direction have been made by
\citet{maziarzReevaluatingRetrosynthesisAlgorithms2024a} who list some best practices relevant and
implemented a software library for standardized testing which also includes multi-step planning
methods. We think that better established standard in this context could establish consensus models
for rating synthesizability in the de novo molecular design. Furthermore, we believe it is important
to acknowledege the limitations of a purely data-driven ML approach to retrosynthesis modelling as
pointed out in \citep{strieth-kalthoffArtificialIntelligenceRetrosynthetic2024}, as lots of chemical
knowledge is available outside of the training data.

Overall we think that both fields would benefit from more collaboration between machine learning
researchers and chemists, in order to better identify practical problems
\citep{benderArtificialIntelligenceDrug2021} and to find ways to solve them.


