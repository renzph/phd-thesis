\chapter{Conclusion and Outlook\label{chap:conclusion}} The work in this thesis has focused on
advancing the application of generative models in drug discovery, concentrating on two main aspects:
Firstly, we identified limitations in the evaluation of generative models for de novo molecular
design, and proposed ways to make evaluation more informative and relevant to practical
applications. Secondly we introduced a novel template-based model for retrosynthesis prediction that
exceeds the performance of existing methods, performing particularly well on rare reaction
templates.

%TODO: This can be more compact and I can add novelty and overestimation as consequence
In the first part of this thesis, we showed how an established benchmark for
distribution-learning models cannot differentiate complex models from trivial baseline generators,
highlighting the need for more informative evaluation metrics. Furthermore, we introduced control
scores as a diagnostic tool to identify overfitting and biases when optimizing \ac{ML}-based scoring
functions in goal-directed molecular generation. We studied a range of generative models
using the control scores, and found that generated molecules indeed show biases towards the
training data and that the generators were able to overfit to artifacts of the scoring functions.


In the second part of this thesis we introduced a diversity-based benchmark for goal-directed
molecule generators. This benchmark measures the performance of goal-directed generators in finding
diverse high-scoring molecules under controlled compute constraints, leading to a principled
evaluation. We used this benchmark to test a range of established generative models, after adapting
them for the diverse optimization setting, leading to the most comprehensive benchmark for diverse
molecule optimization. We found that SMILES-based autoregressive models performed best,
outperforming other models, such as genetic algorithms or GFlowNets.

The third and last part of this thesis introduced a novel template-based model for retrosynthesis
prediction based on Modern Hopfield Networks. This model leverages a multi-modal approach that
combines reaction templates and target molecules. Our model reaches state-of-the-art performance
while maintaining significantly lower computational costs in comparison to existing methods. The model
is able to generalize over reaction templates and performs particularly well on rare ones when
compared to existing methods.

The control scores have since been further investigated in \citep{turkMolecularAssaysSimulator2022}.
We join them in their emphasis that while the control scores can identify the presence of biases,
they cannot establish their absence as the optimization and control scores might be highly
correlated, even outside the applicability domain of the models.
\citet{thomasComparisonStructureLigandbased2021} showed that optimizing structure-based scoring
functions leads to more diverse and novel molecules compared to \ac{ML}-based scoring functions.
This further supports the existence of a bias towards active compounds in the training data when
optimizing \ac{ML}-based scoring functions.

Going forward we see more comprehensive benchmarks for goal-directed generators as a key aspect to
advance de novo molecular design. Despite significant efforts by others
\citep{brownGuacaMolBenchmarkingModels2019,gaoSampleEfficiencyMatters2022,gaoSynthesizabilityMoleculesProposed2020,thomasMolScoreScoringEvaluation2024}
and the work in this thesis, there is a lack of unifying these efforts. One challenge is the
establishment of relevant scoring functions that better reflect the difficulties in real-world drug
discovery projects \citep{fromerComputeraidedMultiobjectiveOptimization2023} without suffering from
the problems observed in \Cref{sec:failure-modes} and \citep{lyuModelingExpansionVirtual2023}. These
benchmarks should also better incorporate whether the chemistry of the generated molecules is
reasonable \citep{thomasReevaluatingSampleEfficiency2022} and diverse. Furthermore the benchmarks
should be run at scales that are more comparable to real-world drug discovery projects, given that
relative performance of the models can be influenced by the available compute budget. Finally, we
think that evaluating synthesizability of the generated molecules is a crucial aspect, which is a
challenge in its own right. While being a significant effort, we believe the establishment of such a
benchmark would spur the development of more practically relevant generative models.

In the field of retrosynthesis prediction, we also believe there is a need for better evaluation
strategies. Efforts in this direction have been made by
\citet{maziarzReevaluatingRetrosynthesisAlgorithms2024a} who list some best practices for evaluation
and implemented a software library for standardized testing which also includes multi-step planning
methods. We think that progress in this field could also lead to consensus models for rating
synthesizability in the de novo molecular design. Furthermore, we believe it is important to
acknowledege the limitations of a purely data-driven ML approach to retrosynthesis modelling as
pointed out in \citep{strieth-kalthoffArtificialIntelligenceRetrosynthetic2024}, as lots of useful
chemical knowledge is available outside of reaction databases.

Overall we think that both fields would benefit from increased collaboration between machine
learning researchers and chemists, to better identify practical problems
\citep{benderArtificialIntelligenceDrug2021} and to find ways to solve them.


