\chapter{Conclusion and Outlook\label{chap:conclusion}} The work in this thesis has focused on
advancing the application of generative models in drug discovery, concentrating on two main aspects:
Firstly, we identified limitations in the evaluation of generative models for de novo molecular
design, and proposed ways to make evaluation more informative and relevant to practical
applications. Secondly we introduced a novel template-based model for retrosynthesis prediction that
exceeds the performance of existing methods, performing particularly well on rare reaction
templates.

In the first part of this thesis, we showed how established ways of evaluating distribution-learning
models cannot differentiate complex models from trivial baseline generators, highlighting the need
for more informative evaluation metrics. Furthermore, we showed how goal-directed generative models
can overfit to machine learning based scoring functions and exhibit biases towards the training
data. We proposed control scores as a diagnostic tool to identify overfitting and biases in
goal-directed molecular generators. The control scores have been further investigated in more detail
by others \citep{turkMolecularAssaysSimulator2022} and it is worth pointing out that that while they
can detect overfitting, they cannot establish its absence. Similar findings have been reported by

The second part of this thesis introduced a diversity-based benchmark for goal-directed molecule
generators aiming at measuring their performance in generating diverse high-scoring molecules.
This benchmark addresses the shortcomings of previous benchmarks by adressing the issues of
inadequate diversity measures, non-standardized compute budgets, and lack of model adaptation to the
diverse optimization setting. We used this benchmark to evaluate a range of generative models
comparing them in a meaningful way.

The last part of this thesis introduced a novel template-based model for retrosynthesis prediction
based on Modern Hopfield Networks. This model leverages a multi-modal approach that combines
reaction templates and target molecules. Our model reaches state-of-the-art performance while
maintaining a significantly lower computational cost compared to existing methods. The model is able
to generalize over reaction templates and performs particularly well on rare reaction templates.

While our work has advanced evaluation of generative models for de novo design, it is hard to
translate current benchmark results into practical utility. Many of the scoring functions used in
current studies are still rather simple one-dimensional functions that do not capture the the
complexities of real-world drug discovery projects. We believe there is a need for more
comprehensive benchmarks that better take the synthesizability, chemistry and novelty of the generated
molecules into account. Furthermore, the model evaluation should move in a direction that better
reflects the scale/resources in real-world drug discovery projects instead of being contrained to
short algorithm runtimes.
% While the field has seen many advances it is hard to pinpoint how
% much practical value these advances have brought to the field
% \citep{benderArtificialIntelligenceDrug2021}.

Similarly, we think that there is a need for more holistic benchmarks of single-step retrosynthesis
prediction models. While single-step performance is an important, we share the view of
\citet{maziarzReevaluatingRetrosynthesisAlgorithms2024a} that there is a need for more standardized
benchmarks that better reflect the performance aspects of single-step methods that are relevant in
multi-step retrosynthesis planning. Furthermore, we believe it is important to acknowledege the
limitations of a pure ML approach based on reaction databases as pointed out in
\citep{strieth-kalthoffArtificialIntelligenceRetrosynthetic2024}. Those limitations are owed to the
quality of the data and the fact that some information needed for robust models is missing from current
reaction databases.

Overall we think that both fields would benefit from more collaboration between machine learning
researchers and chemists, in order to better identify practical needs and to find
computational solutions to these needs.


