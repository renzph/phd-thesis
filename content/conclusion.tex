\chapter{Conclusion and Outlook\label{chap:conclusion}}

The work in this thesis has focused on advancing the application of generative models in drug
discovery, concentrating on two main aspects: Firstly, we identified limitations in the evaluation
of generative models for de novo molecular design, and proposed improvements to make evaluation more
informative and relevant to practical applications. Secondly we introduced a novel template-based
model for retrosynthesis prediction.

In the first part of this thesis, we showed how an established benchmark for distribution-learning
models cannot differentiate complex models from trivial baseline generators, highlighting the need
for more informative evaluation metrics. Furthermore, we introduced control scores as a diagnostic
tool to identify overfitting and biases when optimizing \ac{ML}-based scoring functions in
goal-directed molecular generation. We used these control scores to study a range of generative
models and found that the generated molecules show biases towards already known high-scoring
molecules, and that the generators overfit to artifacts of the scoring functions leading to an
overestimation of performance.

In the second part of this thesis, we introduced a benchmark measuring the performance of
goal-directed generators in finding diverse high-scoring molecules under controlled compute
constraints. We used this benchmark to test a range of established generative models, after adapting
them to the diverse optimization setting. We found that there are large differences in the
performance of the tested models and that the type of compute constraint has a significant impact on
the ranking of the models. In our tests SMILES-based autoregressive models performed best,
outperforming other models, such as genetic algorithms or GFlowNets.

In the third and last part of this thesis, we introduced a novel template-based model for
retrosynthesis prediction based on Modern Hopfield Networks. Our approach allows for generalization
over reaction templates and improves performance, particularly for rare templates. Our model reaches
state-of-the-art performance on the USPTO-50k dataset, while maintaining significantly lower
computational costs in comparison to other methods.

The control scores have since been further investigated in \citet{gendreauMolecularAssaysSimulator2023},
basing their scoring functions on an artificial oracle defining the ground truth. We join them in
their emphasis that while a deviation of the control scores from the optimization score can identify
the presence of biases, the converse need not be true. The model control score may be highly
correlated with the optimization score, even when the ground truth differs. Furthermore, the data
control score may be highly correlated with the optimization score if the data splits they are based
on are highly similar \citep{langevinExplainingAvoidingFailure2022}. Both of these scenarios lead
to cases where the control scores are not able to detect biases in the generated molecules.

Related findings were made in \citet{thomasReevaluatingSampleEfficiency2022}, in which the authors
showed that optimizing structure-based scoring functions leads to more diverse and novel molecules
compared to \ac{ML}-based scoring functions. This further supports the existence of a bias towards
high-scoring molecules in the training data when optimizing \ac{ML}-based scoring functions.
\citet{lyuModelingExpansionVirtual2023} and \citet{wuIdentifyingArtifactsLarge2024} observed
molecules that cheat structure-based scoring functions. They found that these molecules dominate the
top-ranked molecules when performing large-scale virtual screening runs. This underlines our broader
point of molecule optimization methods exploiting unwanted artifacts of computational scoring
functions.

Going forward we see more comprehensive benchmarks for goal-directed generators as a key aspect to
advance the field. Despite significant efforts by others
\citep{brownGuacaMolBenchmarkingModels2019,polykovskiyMolecularSetsMOSES2020,gaoSynthesizabilityMoleculesProposed2020,gaoSampleEfficiencyMatters2022,thomasMolScoreScoringEvaluation2024}
and the work in this thesis, there is a lack of unifying these efforts. One challenge is the
establishment of relevant scoring functions that better reflect the difficulties in real-world drug
discovery projects \citep{fromerComputeraidedMultiobjectiveOptimization2023} without suffering from
the problems observed in \Cref{sec:failure-modes} and
\citep{lyuModelingExpansionVirtual2023,wuIdentifyingArtifactsLarge2024}. These benchmarks should
also better incorporate whether the chemistry of the generated molecules is reasonable
\citep{thomasReevaluatingSampleEfficiency2022} and diverse. Furthermore, the benchmarks should be
run at scales that are more comparable to real-world drug discovery projects, to better reflect
practical scenarios. Finally, we think that evaluating synthesizability of the generated molecules
is a crucial aspect, which is a challenge in its own right. While being a significant effort, we
believe the establishment of such a combined benchmark would spur the development of more
practically relevant generative models.

In the field of retrosynthesis prediction, we also believe there is a need for better evaluation
strategies. Efforts in this direction have been made by
\citet{maziarzReevaluatingRetrosynthesisAlgorithms2024}, who define best practices for evaluation
and implemented a software library for standardized testing, including multistep planning methods.
We think that progress in this field could also lead to consensus models for rating synthesizability
in the de novo molecular design. Furthermore, we believe it is important to acknowledge the
limitations of a purely data-driven ML approach to retrosynthesis modelling as pointed out in
\citet{strieth-kalthoffArtificialIntelligenceRetrosynthetic2024}, as some of the data required for
robust retrosynthesis models is missing in current reaction databases.

In general, we think that the field would benefit from increased collaboration between machine
learning researchers and chemists, to better identify practical problems
\citep{benderArtificialIntelligenceDrug2021} and to find ways to solve them.


