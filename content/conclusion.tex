\chapter{Conclusion and Outlook\label{chap:conclusion}}
The work in this thesis has focused on
advancing the application of generative models in drug discovery, concentrating on two main aspects:
Firstly, we identified limitations in the evaluation of generative models for de novo molecular
design, and proposed ways to make evaluation more informative and relevant to practical
applications. Secondly we introduced a novel template-based model for retrosynthesis prediction
that matches or exceeds the performance of existing methods, performing particularly well on rare
reaction templates.

In the first part of this thesis, we showed how established ways of evaluating distribution-learning
models cannot differentiate complex models from trivial baseline generators. We also showed how
goal-directed generative models used to optimize machine learning-based scoring functions, can
overfit to the scoring function and exhibit biases towards already known high scoring molecules
contained in the training data.

The second part of this thesis introduced a diversity-based benchmark for goal-directed molecule
generators. This benchmark addresses the shortcomings of previous benchmarks by adressing the issues
of inadequate diversity measures, non-standardized compute budgets, and lack of model adaptation to
the diverse optimization setting. We used this benchmark to evaluate a range of generative models
comparing them in a meaningful way.

The last part of this thesis introduced a novel template-based model for retrosynthesis prediction
based on Modern Hopfield Networks. This model leverages a
multi-modal approach that combines reaction templates and target molecules. Our model is able to
generalize over reaction templates and performs particularly well on rare templates. We showed that
our model matches or exceeds the performance.



In conclusion, our work provides insights into the capabilities and limitations of current generative
models for molecules and proposes novel evaluation strategies. Additionally, our contributions in
retrosynthesis prediction enable more accurate computer-aided synthesis planning.
We hope that our work will help to accelerate the drug discovery pipeline and facilitate the development
of novel pharmaceutical treatments.