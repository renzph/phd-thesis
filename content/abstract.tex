% !TeX encoding = UTF-8
% !TeX root = MAIN.tex

{%
\selectlanguage{english}
\chapter*{Abstract}
In recent years the use of generative models in drug discovery has seen increased attention, as
novel deep learning architectures have shown great flexibility in generating molecular structures.
Two main applications of generative models in drug discovery are de novo molecular design and
retrosynthesis prediction. In this thesis, we investigate the limitations of current evaluation
metrics of generative models in de novo molecular design and introduce novel evaluation strategies.
In addition, we introduce a novel approach for retrosynthesis prediction.

The first part of this thesis focuses on observed failure modes in the evaluation of generative
models for de novo molecular design. In particular we show that commonly used metrics used to
evaluate distribution-learning models are not sufficient to differentiate complex models from trivial
baseline generators. Secondly, we show how goal-directed molecular generators can overfit
to machine learning-based objective functions, leading to biases towards the training data and
overestimation of the model's performance.

The second part introduces a diversity-based benchmark for goal-directed molecule generators,
which measures the ability of a model to generate diverse sets of molecules with desired properties.
Previous studies on diverse molecule optimization have been limited by inadequate diversity
measures, non-standardized compute budgets, and lack of model adaptation to diverse optimization
settings. Our benchmark addresses these shortcomings, providing a standardized framework for
evaluating diverse, goal-directed molecule generators and enabling fair model comparisons.

The third part of this thesis focuses on retrosynthesis prediction a crucial task in computer-aided
synthesis planning (CASP). We propose a novel template-based retrosynthesis prediction model based
on Modern Hopfield Networks. Our model takes both the target molecule and the reaction templates as
input, which allows it to generalize over reaction templates, which improves performance,
particularly on rare templates. Our model achieves state-of-the-art performance while maintaining a
significantly lower computational cost compared to existing methods.

Through our work, we provide insights into the capabilities and limitations of evaluation of
generative models for molecules while proposing novel evaluation strategies. Additionally, our
contributions in retrosynthesis prediction enables more accurate computer-aided synthesis planning.
Collectively, these advances have the potential to accelerate the drug discovery pipeline and
facilitate the development of novel pharmaceutical treatments.
}%