% !TeX encoding = UTF-8
% !TeX root = MAIN.tex

{%
\selectlanguage{english}
\chapter*{Abstract}
In recent years the use of generative models in drug discovery has seen a surge, as novel deep
learning architectures have shown great flexibility in generating molecular structures. However, the
evaluation of generative models is challenging as generative tasks are have non-unique solutions and
rating the relevance of the generated molecules can be subjective. This led to biased evaluations
and use of metrics not representative of a practical value. In this thesis, we propose new
evaluation metrics and benchmarks for generative models in drug discovery. Another focus of this
work is the application of generative models to retrosynthesis prediction, a crucial task in
computer-aided synthesis planning (CASP). 

The first part of this thesis focuses on observed failure modes in the evaluation of generative
models for de novo molecular design. In particular we show that commonly used metrics used to
evaluate distribution-learning are not sufficient to differentiate complex models from trivial
baseline generators. Secondly, we show how generative models applied to molecular optimization can
overfit to machine learning-based scoring functions, leading to biased evaluations. We propose new
evaluation metrics that are more robust to these failure modes and provide a more accurate
representation of the model's performance. 

In the second part of this thesis we introduce a diversity-based benchmark of goal-directed molecule
generators. Finding a diverse range of high-scoring compounds is important for the success of drug
discovery projects as some of the found candidates are expected to fail in downstream experiments.
However, previous comparative studies in diverse molecule optimization suffer from inadequate
diversity measures, lack of standardized compute budgets and missing adaption of the generative models
to the diverse optimization setting. Our benchmark addresses these issues and provides a standardized
evaluation framework for diverse, goal-directed molecule generators, enabling a fair comparison of
different models.

The third part of this thesis focuses on retrosynthesis prediction a crucial task in computer-aided
synthesis planning (CASP). We propose a novel template-based retrosynthesis prediction model based
on Modern Hopfield Networks. Our model takes both the target molecule and the reaction templates 
as input, which allows it to generalize over reaction templates, which improves 
performance, particularly on rare templates. Our model achieves state-of-the-art performance on the USPTO-50k dataset.
while maintaining a significantly lower computational cost compared to existing methods. 

Through our work, we provide insights into the capabilities and limitations of
current generative models for molecules while proposing novel evaluation
strategies. Additionally, our contributions in retrosynthesis prediction enable
more accurate computer-aided synthesis planning. Collectively, these advances
have the potential to accelerate the drug discovery pipeline and facilitate the
development of novel pharmaceutical treatments.
}