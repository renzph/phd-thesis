% !TeX encoding = UTF-8
% !TeX root = MAIN.tex

{%
\selectlanguage{english}
\chapter*{Abstract}
In recent years, the use of generative models in drug discovery has gained significant attention,
spurred by advances in deep learning. Deep neural networks have been applied to solve various tasks
requiring the generation of novel molecular structures. This thesis focuses on two major
applications of generative models: de novo drug design and retrosynthesis prediction. We investigate
limitations of existing evaluation strategies for generative models in de novo drug design and
propose improved methodologies. Furthermore, we present a novel method for retrosynthesis
prediction.

The first part of this thesis examines observed failure modes in evaluating generative models for de
novo molecular design. We demonstrate that commonly used metrics used to evaluate
distribution-learning models are not sufficient to differentiate complex models from trivial
baseline generators. Furthermore, we show how goal-directed molecule generators can overfit to
machine learning-based objective functions, leading to biases towards the training data causing a
lack of novelty and overestimation of model performance.

The second part introduces a diversity-based benchmark for goal-directed molecule generators, which
quantifies a model's ability to generate diverse sets of molecules with desired properties. Previous
studies on diverse molecule optimization have been limited by inadequate diversity measures,
non-standardized compute budgets, and lack of model adaptation to diverse optimization settings. Our
benchmark addresses these shortcomings, providing a standardized framework for evaluating
goal-directed molecule generators in diverse optimization settings, enabling fair model
comparisons.

% Given a target molecule, our model predicts which reaction templates lead to
% feasible reactions when applied to the target molecule, thus generating sets of reactants that can
% be used to synthesize the target molecule.
In the third part, we propose a novel template-based retrosynthesis prediction model based on Modern
Hopfield Networks. Our model learns representations of the target molecule and reaction templates,
and how to associate them. This enables generalization over reaction templates and improves
performance, particularly for rare templates. Our model achieves state-of-the-art performance while
maintaining significantly lower computational costs compared to existing methods.

Through our work, we provide insights into the limitations of current evaluation strategies for
generative models for molecules while proposing new ones. Additionally, our
contribution in retrosynthesis prediction enables more accurate computer-aided synthesis planning.
Collectively, these advances have the potential to accelerate the drug discovery pipeline and
facilitate the development of novel pharmaceutical treatments.
}%