% !TeX encoding = UTF-8
% !TeX root = MAIN.tex

{%
\selectlanguage{english}
\chapter*{Abstract}
The discovery of novel pharmaceutical treatments is a major driver of quality of
life improvements. However, the traditional drug discovery process involving
synthesis and experimental testing is extremely expensive, time-consuming, and
risky for patients. Computer-aided drug discovery (CADD) leveraging machine
learning (ML) techniques has the potential to accelerate this process
significantly. This thesis focuses on two key areas where ML can help accelerate
drug discovery: generative models for molecules and computer-aided synthesis
planning (CASP).

Generative models aim to efficiently explore the vast chemical space of
drug-like molecules to discover novel compounds satisfying desired property
profiles. However, evaluating these models is challenging due to their
generation of complex, structured molecular outputs. We address this issue by
proposing new evaluation metrics, benchmarks, and uncovering potential failure
modes that can lead to overfitting and biased generation.

For CASP, accurate retrosynthesis prediction models are crucial for suggesting
viable synthesis routes. We propose a novel Hopfield network-based approach for
template-based retrosynthesis prediction. Our model leverages structural
information to improve generalization, especially for rare templates and even
unseen reaction types, while being significantly more efficient than existing
methods.

Through our work, we provide insights into the capabilities and limitations of
current generative models for molecules while proposing novel evaluation
strategies. Additionally, our contributions in retrosynthesis prediction enable
more accurate computer-aided synthesis planning. Collectively, these advances
have the potential to accelerate the drug discovery pipeline and facilitate the
development of novel pharmaceutical treatments.

}
